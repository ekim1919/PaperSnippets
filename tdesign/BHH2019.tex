% Created 2020-09-06 Sun 22:05
% Intended LaTeX compiler: pdflatex
\documentclass[a4paper]{article}
\usepackage[utf8]{inputenc}
\usepackage[T1]{fontenc}
\usepackage{graphicx}
\usepackage{grffile}
\usepackage{longtable}
\usepackage{wrapfig}
\usepackage{rotating}
\usepackage[normalem]{ulem}
\usepackage{amsmath}
\usepackage{textcomp}
\usepackage{amssymb}
\usepackage{capt-of}
\usepackage{hyperref}
\usepackage[margin=0.5in]{geometry}
\usepackage{amsthm}
\author{Edward Kim}
\date{\today}
\title{BHH2019}
\hypersetup{
 pdfauthor={Edward Kim},
 pdftitle={BHH2019},
 pdfkeywords={},
 pdfsubject={},
 pdfcreator={Emacs 26.3 (Org mode 9.4)}, 
 pdflang={English}}
\begin{document}

\maketitle
\tableofcontents


\section{Major Findings to Investigate/Understand}
\label{sec:orgac81dc7}
\begin{itemize}
\item Random Haar Unitary requires exponential number of two-qubit gates and random bits.
This makes it computationally infeasible in practice to select such a unitary according to Haar measure.
\href{https://arxiv.org/pdf/quant-ph/9508006.pdf}{Knill. Approximations by Quantum Circuits}

\item Best result for a some time was that polynomial random quantum circuits are approximate 2-designs.
\item A unitary chosen under Haar measure can be thought of sampling from a uniform distribution over all unitaries in the unitary group.
A quantum expander's rapid mixing time (polynomial in \(t\) for a \(t\) -design) means that we can rapidly converge to such a unitary.

\item Folklore result: An overwhelming number of pure quantum states on n qubits is indistinguihable from the maximally mixed state if restricted to measurements implemented with subexponential-sized quantum circuits.
\href{https://arxiv.org/pdf/0812.3001.pdf}{Are random pure states useful for quantum computation?}

\item The proof of the main result concerns bounding the mixing time by the spectral gap (Cheeger-like inequality)
\href{https://arxiv.org/pdf/0910.0913.pdf}{Convergence rates for arbitrary statistical moments of random quantum circuits}
\item Previous papers have weaker results and are probably worth taking a look for learning techniques behind proofs.
\href{https://arxiv.org/pdf/0811.2597.pdf}{Efficient Quantum Tensor Product Expanders and k-designs}
\href{https://link.springer.com/content/pdf/10.1007/s00220-009-0873-6.pdf}{Random Quantum Circuits are Approximate 2-designs}
\end{itemize}

\section{Definitions}
\label{sec:org3570d46}
\begin{itemize}
\item The definition of \emph{Quantum Tensor Product Expander} \(g(\upsilon,t,\lambda)\), is a distribution \(\upsilon\) over \(\mathbb{U}(N)\) such that
$$ g(\upsilon, t) = || \int_{\mathbb{U}(N)} U^{\otimes t} \otimes (U^*)^{\otimes t} \upsilon(dU) - \int_{\mathbb{U}(N)} U^{\otimes t} \otimes (U^*)^{\otimes t} \mu(dU)||_{\infty} \leq \lambda $$
where \(\mu\) is the Haar measure over Lie group \(\mathbb{U}(N)\)

We can then think about a local random circuit on n qudits as a random walk such that we pick some index i and some unitary \(U_{i,i+1} \in \mathbb{U}(d^2)\) and we apply \(U_{i.i+1}\) on the qubits \(i\) and \(i+1\). This gives us a distribution over \(\mathbb{U}(d^n)\).
\item Let \(\mu\) be a distribution on \(\mathbb{U}(4)\). Suppose that for any open ball \(S \subset \mathbb{U}(4)\) there exists a positive integer \(\ell\) such that \(\mu^{\star}(S) > 0\) where $$\mu^{\star} = \int_{S} \delta_{U_1 \dots U_{\ell}} d_{\mu}(U_1)\dots d_{\mu}(U_{\ell}) $$

\item Define \(\hat{G}\)
\end{itemize}

\section{Statement of Major Theorems}
\label{sec:org99a91a4}
\begin{itemize}
\item Let \(\mu\) be be a 2-copy gapped distribution and \(W\) be a random circuit sampled from \(\mu\) be choosing \(t\) circuits and choosing a random pair of qudits (\({n \choose 2}\) in all). Then \(t(n) \in \Omega(n(n+1 + \log(1/\epsilon)))\). Then \(G_W\) is an \(\epsilon\)-approximate 2-design.

This shows that any distribution \(\mu\) which is 2-gapped over \(\mathbb{U}(d)\) produces an \$\(\epsilon\)\$-approximate 2-design in ``efficient time'' (polynomial in \(n\) and log factor \(1/\epsilon\)).

\item The crux of the proof above is to show that the expected coefficients  \(\gamma_W(p_1,p_2) = \mathbb{E}_W \gamma_W(p_1,p_2)\) where the expectation is taken over all random circuits of length \(t\) converge to that of the uniform distribution:

For any initial state \(\rho = \frac{1}{2^n} \sum_{p_1, p_2} \gamma_0(p_1, p_2) \sigma_{p_1}\otimes \sigma_{p_2}\) where \(p_1,p_2\) are \(n\) -strings \(\{0,\dots,d-1\}^n\) denoting the \(n\) -tensor product Pauli operators corresponding to
\(p_1,p_2\), there exists a constant \(C\) such that for any \(\epsilon > 0\),
$$ \sum_{p_1,p_2 \; p_1p_2 \neq 00} \left( \gamma_t(p_1,p_2) - \delta_{p_1p_2}\frac{\sum_{p \neq 0} \gamma_0(p,p)}{4^n-1} \right)^2 \leq \epsilon$$
for \(t \geq C(n(n+ \log{1/\epsilon}))\) i.e \(t(n) \in \Omega(n(n+ \log{1/\epsilon}))\)

We see that the lemma claims that all off-diagonal coefficients \(\gamma_t(p_1,p_2), \; p_1\neq p_2\) will vanish as \(t \rightarrow \infty\) while \(\gamma_t(p_1,p_2), \; p_1 = p_2\) converges to some uniform distribution dependent on the initial coefficients \(\gamma_0(p,p)\). Note that \(\gamma_0(p,p)\) can have \emph{negative} values. Furthermore, \(\sum_p \gamma(p,p)\) does not necessarily have to sum to one, initially precluding us from considering these coefficients as a probability distribution. We introduce another lemma which allows us to perform Markov chain analysis when \(\gamma_0(p,p) \geq 0\) and \(\sum_p \gamma_p(p,p) = 1\).

\item One property we have to check is that sampling from an universal distribution on \(\mathbb{U}(d)\) and applying a sufficiently long random circuit to an initial set will converge to the Haar distribution on \(\mathbb{U}(d)\) in the first place.

\begin{itemize}
\item Question: To prove k-copy gappedness, why is sufficient to prove that \(||G_{\mu} - G_{\mathbb{U}(d)}||_{\infty} < 1\) ? Here, \(G_{\mu} = \mathbb{E}_U U^{\otimes k} \otimes (U^*)^{\otimes k}\) where the expectation is taken over universal distribution \(\mu\) over \(\mathbb{U}(d)\) and \(G_{\mathbb{U}(d)}\) is taken over Haar measure on \(\mathbb{U}(d)\).

\item The following theorem shows that \(||G_{\mu} - G_{\mathbb{U}(d)}||_{\infty} < 1\):

Let \(\mu\) be a distribution over \(\mathbb{U}(d)\). Then all eigenvectors of \(G_{\mathbb{U}(d)}\) of eigenvalue one are also eigenvectors of \(G_{\mu}\) with eigenvalue one. If \(\mu\) is universal, then \(\mu\) is also k-copy gapped.
\end{itemize}
\end{itemize}



\begin{itemize}
\item Since 2-qubit gates are universal for \(\mathbb{U}(2^n)\), creating random circuits converges to the Haar distribution. Now if we let \(\hat{G}\) be the transition matrix comprised as follows: Let \(T(p) = \mathbb{E}_U U^{\otimes k} \sigma_{p_1} \dots \sigma_{p_k} (U^\dagger)^{\otimes k}\) for some \(k \in \mathbb{Z}^+\) and \(0 \leq p_i \leq d^2-1\) for all \(i\). In other words, \(T(p)\) represents the expected state over all unitaries sampled via some distribution \(\mu\) over \(\mathbb{U}(d)\). Note that is this a general construction for arbitrary k-moments and \(\mathbb{U}_d\). For our case, we will be interested in the case where \(k=2\). Since we can reexpress \(T(p)\) in terms of the \(\sigma_p\) basis above:
$$ T(p) = \sum_{q} \hat{G}(q;p) \sigma_{q_1}\dots\sigma_{q_k} $$
so
$$ \hat{G}(p;q) = d^{-k} tr(\sigma_q T(p)) $$
We can now consider \(\hat{G}\) as a matrix of coefficients describing the transformation of basis vectors \(\sigma_{p_1}\otimes \dots \otimes \sigma_{p_k}\).

\begin{itemize}
\item Some facts about \(\hat{G}\):
\begin{enumerate}
\item \(\hat{G}\) is a linear combination of permutation operators \(P_{\pi} \in S_k\).
This follows from the Schur-Weyl Duality since \(T(p)\) commutes with \(U^{\otimes k}\) for any unitary \(U\).
\item \(\hat{G}\) is symmetric.
This follows from the cyclic property of the trace.
\item The permutation operators are eigenvectors of eigenvalue 1 of \(T(p)\) i.e
$$ T(p)P_{\pi} = P_{\pi} $$ or $$ \sum_{q} \hat{G}(q;p) tr(\sigma{q_1} \otimes \dots \otimes \sigma_{q_k} P_{\pi}) = tr(\sigma_{q_1} \otimes \dots \otimes \sigma_{q_k} P_{\pi}) $$

\item Using the above three observations, we conclude that \(\hat{G}\) is a projector i.e \(\hat{G}^2 = \hat{G}\).
\end{enumerate}
\item A crucial part of the argument of proving the above lemma will be to consider the transitions:
$$ \gamma_{t+1}(p) = \sum_{i\neq j} \frac{1}{n(n-1)}\sum_{q} \hat{G}^{ij}(q)\gamma_t(q) $$
which calculates the \emph{expected} coefficents for n-string \(q\) in a similar way we calculate probability distributions for Markov chains.
\end{itemize}

\item Don't really understand the definition of \(g_t(p,p;q,q)\).

\item 
\end{itemize}

\section{Questions to consider}
\label{sec:org6278c85}

\begin{itemize}
\item Is it worth learning the tensor-product expander constructions now? Maybe I should make a short article about it?
Yes. It would be friutful to learn some relevant Quantum Complexity Theory on the sid
\end{itemize}
\end{document}
